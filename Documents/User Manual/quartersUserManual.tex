%&pdflatex
\documentclass[12pt]{article}


\usepackage{graphicx}
\graphicspath{ {images/} }
\usepackage{colortbl}
\usepackage{xr}
\usepackage{longtable}
\usepackage{xfrac}
\usepackage{tabularx}
\usepackage{booktabs}
\usepackage{hyperref}
\usepackage{xcolor} % for different colour comments
\usepackage{fullpage}
\newcounter{rowcount}
\setcounter{rowcount}{0}
\usepackage{tikz}
\usetikzlibrary{shapes,arrows}


%% Diagram formatting
\tikzstyle{block} = [draw, rectangle, minimum height=1.5em, minimum
width=2em, text centered]
\tikzstyle{arrow} = [thick,-,>=stealth]


\hypersetup{
    bookmarks=true,         % show bookmarks bar?
    colorlinks=true,        % false: boxed links; true: colored links
    linkcolor=blue,        % color of internal links (change box color with linkbordercolor)
    citecolor=green,        % color of links to bibliography
    filecolor=magenta,      % color of file links
    urlcolor=cyan           % color of external links
}

%% Comments
\newif\ifcomments\commentstrue
\ifcomments
\newcommand{\authornote}[3]{\textcolor{#1}{[#3 ---#2]}}
\newcommand{\todo}[1]{\textcolor{red}{[TODO: #1]}}
\else
\newcommand{\authornote}[3]{}
\newcommand{\todo}[1]{}
\fi
\newcommand{\wss}[1]{\authornote{magenta}{SS}{#1}}
\newcommand{\ds}[1]{\authornote{blue}{DS}{#1}}
\newcommand{\kly}[1]{\authornote{green}{KL}{#1}}
\newcommand{\cc}[1]{\authornote{orange}{CC}{#1}}

%%%%%%%%%%%%%%%%%%%%%%%%%%%%%

\begin{document}

\title{User Manual}
\author{Team 6\\ \\James Anthony (anthonjb)\\ Wenqiang Chen (chenw25)\\ Carolyn Chong
(chongce)\\ Kevin Ly (lyk2)}
\date{\today}

\maketitle

\pagebreak

\tableofcontents
\listoffigures

\section*{Revision History}
\begin{tabular}{|c|c|}
\hline
\textbf{Date}  & \textbf{Comments} \\ \hline
February 28, 2016 & Revision 0 \\ \hline
\end{tabular}

\pagebreak

%%%%%%%%%%%%%%%%%%%%%%%%%%%%%

%%%%%%%%%%%%%%%%%%%%%%%%%%%%%% Introduction
\section{Introduction}

Words

%%%%%%%%%%%%%%%%%%%%%%%%%%%%%% Copyright
\section{Copyright}

%%%%%%%%%%%%%%%%%%%%%%%%%%%%%% About this Manual
\section{About this Manual}

%%%%%%%%%%%%%%%%%%%%%%%%%%%%%%
\section{Installation}

%%%%%%%%%%%%%%%%%%%%%%%%%%%%%% Tasks
\section{Tasks}

\subsection{Registration}
\label{sec:registration}
All user must have a registered account before using Quarters. Registration process is simple and easy, only a valid email and a password is required.\\\\
You can create an account by following these instructions:

\begin{enumerate}
    \item Click on "SIGN UP" button on the top of the homepage, or Click on "LOGIN" button then select "Register" tab
    \item Enter a valid email address and password
    \item Click on "REGISTER NOW"
    \item An email will be sent to the specified email address, which will contain a link for you to activate your account.
\end{enumerate}
\subsection{Login}
All user must be logged in to use any of the Quarters' features. If you do you have an account, please refer to \hyperref[sec:registration]{Registration} to create an account. \\ \\
You can login by following these instructions:
\begin{enumerate}
    \item Click on "LOGIN" on the top of the homepage, or Click on "SIGN UP" button and switch to "Login" tab
    \item Enter your registered email address and password
    \item Click on "LOG IN"
\end{enumerate}
\subsection{House Management}

\subsection{User Profile}

\subsection{House Information}

\subsection{Document Upload}

\subsection{Bulletin Board}

\subsection{Finance}


\subsection{Ticketing}

\subsection{Calendar}

\subsection{Notifications}



%%%%%%%%%%%%%%%%%%%%%%%%%%%%%% Troubleshooting
\section{Troubleshooting}

%%%%%%%%%%%%%%%%%%%%%%%%%%%%%% Frequently Asked Questions
\section{Frequently Asked Questions}

%%%%%%%%%%%%%%%%%%%%%%%%%%%%%% Safety and Precautions

\end{document}
