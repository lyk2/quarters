\documentclass[12pt]{article}


\usepackage{graphicx}
\usepackage{colortbl}
\usepackage{xr}
\usepackage{longtable}
\usepackage{xfrac}
\usepackage{tabularx}
\usepackage{booktabs}
\usepackage{hyperref}
\usepackage{xcolor} % for different colour comments
\usepackage{fullpage}
\newcounter{rowcount}
\setcounter{rowcount}{0}

\hypersetup{
    bookmarks=true,         % show bookmarks bar?
      colorlinks=true,       % false: boxed links; true: colored links
    linkcolor=black,          % color of internal links (change box color with linkbordercolor)
    citecolor=green,        % color of links to bibliography
    filecolor=magenta,      % color of file links
    urlcolor=cyan           % color of external links
}

%% Comments
\newif\ifcomments\commentstrue
\ifcomments
\newcommand{\authornote}[3]{\textcolor{#1}{[#3 ---#2]}}
\newcommand{\todo}[1]{\textcolor{red}{[TODO: #1]}}
\else
\newcommand{\authornote}[3]{}
\newcommand{\todo}[1]{}
\fi
\newcommand{\wss}[1]{\authornote{magenta}{SS}{#1}}
\newcommand{\ds}[1]{\authornote{blue}{DS}{#1}}
\newcommand{\kly}[1]{\authornote{green}{KL}{#1}}
\newcommand{\cc}[1]{\authornote{orange}{CC}{#1}}

%%%%%%%%%%%%%%%%%%%%%%%%%%%%%

\begin{document}

\title{Test Plan for Quarters} 
\author{James Anthony (anthonjb)\\ Wenqiang Chen (chenw25)\\ Carolyn Chong 
(chongce)\\ Kevin Ly (lyk2)}
\date{\today}
	
\maketitle

\pagebreak

\tableofcontents

\section*{Revision History}
\begin{tabular}{|c|c|}
\hline
\textbf{Date}  & \textbf{Comments} \\ \hline
October 21, 2015 & Created first draft. \\ 
\hline
\end{tabular}

\section*{Template}
This document makes use of the Software Test Plan (STP) Template for all of its organization.

\pagebreak

%%%%%%%%%%%%%%%%%%%%%%%%%%%%%

%Acronyms and Definitions
\section{Acronyms and Definitions}
\renewcommand{\arraystretch}{1.2}
\begin{tabular}{l l} 
  \toprule		
  \textbf{Acronym} & \textbf{Description}\\
  \midrule 
  PoC		&Proof of Concept\\
  \bottomrule
\end{tabular}\\


%Plans for Automated Testing
\section{Plans for Automated Testing}


%Plans for Unit Testing
\section{Plans for Unit Testing}


%System Tests
\section{System Tests}

%Template
\cc{Use this as a template. Create a new subsection for each feature to be tested.}\\
\textbf{Test Type:} Structural/Functional/Unit, Static/Dynamic, Manual/Automated. \\
\textbf{Test Factors:} Correctness/Learnability/Maintainability/Reliability. \\
\textbf{Tools Used:} unit testing framework, code coverage metrics, static checkers, automated testing, load testing (like JMeter), etc. \\ 
\textbf{Schedule:} PoC Demo November 16 / Final Demo April 1. \\
\textbf{Team Member Responsible:} \\
\textbf{Methodology:} The "how".

\begin{longtable}{|p{2cm}|p{3cm}|p{5cm}|p{5cm}|}
\hline
\textbf{Test Case}  & \textbf{Initial State} & \textbf{Input} & \textbf{Output} \\ \hline
4. & & &  \\ 
\hline
\end{longtable}



%4.1 User Registration
\subsection{User Registration} \cc{Should I include Google/Facebook?}\\
\textbf{Test Type:} Functional/Static/Dynamic/Manual/Automated. \\
\textbf{Test Factors:} Correctness, Reliability. \\
\textbf{Tools Used:} Selenium, Google reCAPTCHA. \\
\textbf{Schedule:} Begin static testing November 6. Complete manual dynamic tests by PoC Demo November 16. Complete automated dynamic tests by Final Demo April 1. \\
\textbf{Team Member Responsible:} \\
\textbf{Methodology:} The main objective of user registration is to create a user account to be used for login. Users must use a valid email address and pass a user identification procedure. This ensures the user is human and prevents spam and automated scripts from accessing the application and abusing its services. Testing is manual and automated. Manual testing involves people manually going through the registration process in real-time as a user. Automated testing involves systemically attempting SQL injections to test for valid and invalid registrations. Google reCAPTCHA validates that users are legitimate.

\begin{longtable}{|p{2cm}|p{3cm}|p{5cm}|p{5cm}|}
\hline
\textbf{Test Case}  & \textbf{Initial State} & \textbf{Input} & \textbf{Output} \\ \hline
4.1.1 & Registration page. Empty fields. & Valid values entered and passes reCAPTCHA test. & Redirected to registration confirmation page. Correct user values are displayed. User values are correctly stored in database. \\ 
\hline
4.1.2 & Registration page. Empty fields. & Empty field(s). & Stays on the same page. Error message appears. Empty field is highlighted. \\
\hline
4.1.3 & Registration page. Empty fields. & Invalid email address. & Stays on the same page. Error message appears. Email field is highlighted. \\
\hline
4.1.4 & Registration page. Empty fields. & Email address already stored in database. & Stays on the same page. Error message appears. Email field is highlighted. \\
\hline
4.1.5 & Registration page. Empty fields. & Fails reCAPTCHA test. & Stays on the same page. Error message appears. Test field is highlighted. \\
\hline
\end{longtable}



%4.2 User Login
\subsection{User Login}
\textbf{Test Type:} Functional/Static/Dynamic/Manual/Automated. \\
\textbf{Test Factors:} Correctness, Reliability. \\
\textbf{Tools Used:} Selenium. \\
\textbf{Schedule:} Begin static testing November 6. Complete manual dynamic tests by PoC Demo November 16. Complete automated dynamic tests by Final Demo April 1. \\
\textbf{Team Member Responsible:} \\
\textbf{Methodology:} The main objective of user login is to ensure a secure process where only valid users are allowed to enter the application. Testing involves authenticating users against an existing database to determine if they are valid users or not. Testing is manual and automated. Manual testing involves people manually going through the login process in real-time as a user. Automated testing involves systemically attempting SQL injections to test for valid and invalid logins.

\begin{longtable}{|p{2cm}|p{3cm}|p{5cm}|p{5cm}|}
\hline
\textbf{Test Case}  & \textbf{Initial State} & \textbf{Input} & \textbf{Output} \\ \hline
4.2.1 & Login page. Empty username and password fields. & Valid username and password combination. & Redirected to application main page. \\ 
\hline
4.2.2 & Login page. Empty username and password fields. & Invalid username. & Stays on the same page. Error message appears. Fields are highlighted. \\
\hline
4.2.3 & Login page. Empty username and password fields. & Valid username and invalid password. & Stays on the same page. Error message appears. Password field is highlighted. \\
\hline
4.2.4 & Login page. Empty username and password fields. & Empty username and/or password fields. & Stays on the same page. Error message appears. Fields are highlighted. \\
\hline
\end{longtable}



%4.3 Calendar
\subsection{Calendar}
\textbf{Test Type:} Functional/Static/Dynamic/Manual/Automated. \\
\textbf{Test Factors:} Learnability, Reliability. \\
\textbf{Tools Used:} \cc{todo}. \\
\textbf{Schedule:} Begin static testing November 6. Complete manual dynamic tests by PoC Demo November 16. Complete automated dynamic tests by Final Demo April 1. \\
\textbf{Team Member Responsible:} \\
\textbf{Methodology:} The Calendar feature allows users to add/delete events and chores to a shared Calendar between members of a house. This shared Calendar can be synched with a user's personal Calendar. Testing is manual and automated. Manual testing involves a person manually going through the process of adding/deleting an event or chore to the Calendar in real-time as a user, and then checking if those updates are properly synched with the user's personal Calendar. Automated testing involves \cc{todo}. \cc{Unit Testing?}.

\begin{longtable}{|p{2cm}|p{3cm}|p{5cm}|p{5cm}|}
\hline
\textbf{Test Case}  & \textbf{Initial State} & \textbf{Input} & \textbf{Output} \\ \hline
4.3.1 & Calendar page. Empty form. & Add event/chore. Correct values entered in fields. & Form closes. Event/chore is added to database. Event/chore is updated on Calendar. \\ 
\hline
4.3.2 & Calendar page. Empty form. & Add event/chore. Incorrect values entered in fields. & Form remains open. Error message appears. Incorrect fields are highlighted. \\
\hline
4.3.3 & Calendar page. Empty form. & Add event/chore. Empty field(s). & Form remains open. Error message appears. Empty fields are highlighted. \\
\hline
4.3.4 & Calendar page. & Click button to delete event/chore.  & Event/chore is removed from database. Event/chore is no longer displayed on Calendar. \\
\hline
\end{longtable}









\end{document}